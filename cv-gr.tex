%!TEX TS-program = xelatex
\documentclass{mycv}

%%%%% LANGUAGE AND FONTS 
\usepackage{xltxtra,xgreek,fontspec} 			 
\setmainfont{Times New Roman}            % main font
%\setmonofont{Courier New}		 % for commands
\usepackage[greek]{datetime2} 		 % show greek date correctly using \today command
\renewcommand{\today}{\ifcase\month\or%
	Ιανουάριος\or Φεβρουάριος\or Μάρτιος\or% 
	Απρίλιος\or Μάιος\or Ιούνιος\or Ιούλιος%
	\or Αύγουστος\or Σεπτέμβριος\or% 
	Οκτώβριος\or Νοέμβριος\or% 
	Δεκέμβριος\fi\ \number \year}			 % nominative instead of genitive 

\hypersetup{%
  pdftitle={Χρήστος Δαλαμάγκας - Βιογραφικό},
  pdfauthor={Χρήστος Δαλαμάγκας},
  pdfsubject={Βιογραφικό},
  pdfkeywords={Χρήστος Δαλαμάγκας, Christos Dalamagkas}
}

\begin{document}
	\thispagestyle{plain}
	\begin{center}
		\name{Χρήστος Δαλαμάγκας}{Βοηθος Ερευνας}{Μηχανικος Δικτυων (CCNA)}
		\contact{(+30) 698 316 0295}{me@christos.pw}{chris.dal}{https://christos.pw}{linkedin.com/in/cdalamagkas}{cdalamagkas}{0000-0002-0210-5290}
		\centering
		{\bf Τοποθεσία}: Αθήνα, Ελλάδα
	\end{center}
	%
	%\vspace*{-0.5cm}
	%
	\section{Εκπαιδευση}
	
	\begin{EntryDatedLogo}{Πανεπιστήμιο Δυτικής Μακεδονίας}{http://ece.uowm.gr}{2012 -- 2017}{Διπλωμα (Πενταετους φοιτησης), Πρ. Τμημα Μηχανικων Πληροφορικης και Τηλεπικοινωνιων}{-1.25cm}{uowm}{0.6}
		\begin{Itemize}
			%\item Recognized as integrated master degree (level 7 of EFQ) under government gazette 3987/14-9-2018
			%\item Thesis title: "\textit{Design of Market Mechanism for Dynamic Bandwidth Allocation on XG-PON}".
			\item Θεωρία παιγνίων στα δίκτυα XG-PON: \url{https://dspace.uowm.gr/xmlui/handle/123456789/835}.
			\item Βαθμός αποφοίτησης: 8.2/10.
		\end{Itemize}
	\end{EntryDatedLogo}
	
	\section{Επαγγελματικη εμπειρια}
	\begin{EntryDatedLogo}{Δημόσια Επιχείρηση Ηλεκτρισμού Α.Ε. (ΔΕΗ)}{https://www.dei.gr/en}{Μάϊ. 2018 - Τώρα}{Εξωτερικος Συνεργατης - Ερευνητης}{-1.25cm}{dei}{0.6}
	\begin{Itemize}
		\item Ερευνητής σε ευρωπαϊκά προγράμματα: \href{https://cordis.europa.eu/project/id/787011}{SPEAR H2020}, \href{https://cordis.europa.eu/project/id/833955}{SDN-microSENSE H2020}, \href{https://cordis.europa.eu/project/id/957406}{TERMINET H2020}, \href{https://www.jaunty.eu/}{JAUNTY Erasmus+}, \href{https://cordis.europa.eu/project/id/101021936}{ELECTRON H2020}, \href{https://cordis.europa.eu/project/id/101070450}{AI4CYBER HE}, \href{https://cordis.europa.eu/project/id/101070455}{DYNABIC HE}, \href{https://cordis.europa.eu/project/id/101056765}{EV4EU HE}.
		\item Ερευνητικοί τομείς: Industrial IoT, software-defined networking, cybersecurity.
		\item Διαχειριστής συστημάτων στο Εργαστήριο Διαδικτύου των Πραγμάτων και Βιομηχανικών Εφαρμογών του \href{https://innovationhub.dei.gr/el/}{Κέντρου Καινοτομίας της ΔΕΗ}.

	\end{Itemize}
	\end{EntryDatedLogo}

	\vspace*{0.5cm}

	\begin{EntryDatedLogo}{Ελεύθερος Επαγγελματίας - Μηχανικός ICT}{https://christos.pw}{Jun. 2018 - 2023}{Μηχανικος Πεδιου}{-1cm}{engineer}{0.6}
		\begin{Itemize}
			\item Εργολήπτης μηχανικός δικτύων με πιστοποίηση CCNA.
		\end{Itemize}
	\end{EntryDatedLogo}

	\vspace*{0.5cm}	

	%\begin{EntryDatedLogo}{Πανεπιστήμιο Δυτικής Μακεδονίας}{https://ece.uowm.gr}{Μάρ. 2017 - Τώρα}{Βοηθος ερευνας}{-1cm}{uowm}{0.6}
	%	\begin{Itemize}
	%		\item Βοηθός εργαστηρίου στα Δίκτυα Υπολογιστών, Κυβερνοασφάλεια και Προσομοίωση.
	%		\item Προετοιμασία ερευνητικών προτάσεων, υλοποίηση ευρωπαϊκών ερευνητικών προγραμμάτων.
	%		\item Προγραμματιστής (Εφαρμογές ιστού, SDN), DevOps, διαχείριση συστημάτων.
	%	\end{Itemize}
	%\end{EntryDatedLogo}
	
	\vspace*{0.5cm}
	
	\begin{EntryDatedLogo}{ΙΙΕΚ ΑΛΦΑ}{https://iekalfa.gr}{Οκτ. 2018 - Ιουν. 2019}{Δασκαλος}{-1cm}{alfa}{0.6}
		\begin{Itemize}
			\item Δίκτυα Υπολογιστών και Τηλεπικοινωνίες.
			\item Λειτουργικά Συστήματα και Αντικειμενοστραφής Προγραμματισμός (C++).
			\item Εκπαιδευτικό υλικό διαθέσιμο στο \href{https://github.com/cdalamagkas/cs-lectures-gr}{GitHub}.
		\end{Itemize}
	\end{EntryDatedLogo}

	\vspace*{0.5cm}
		
	\begin{EntryDatedLogo}{Πανεπιστήμιο του Brighton}{https://www.brighton.ac.uk}{Ιουλ. - Σεπτ. 2017}{Πρακτικη Erasmus+}{-0.45cm}{brighton}{0.6}
	\end{EntryDatedLogo}

	\vspace*{0.75cm}	

	%\begin{EntryDatedLogo}{IntelliSolutions S.A}{http://intelli-corp.com}{Ιούλ. - Αυγ. 2016}{Μηχανικος Δικτυων και Συστηματων (Ασκουμενος)}{-0.4cm}{intelli}{0.75}
	%\end{EntryDatedLogo}
	%\newpage
	
	\section{Επαγγελματικες Πιστοποιησεις}
	\begin{EntryDatedLogo}{Cisco Certified Network Associate (CCNA)}{https://www.cisco.com/}{\scshape{Ιουλ. 2019}}{}{-0.75cm}{cisco}{0.6}
	\end{EntryDatedLogo}
	\vspace{0.25cm}
	
	\section{Δεξιοτητες}
	\begin{tabular}{m{4.5cm} m{13cm}}\renewcommand{\arraystretch}{2}
		\textbf{Δίκτυα Υπολογιστών}   	& Cisco IOS, Open vSwitch, ArubaOS-Switch, OpenFlow, Python Ryu.\\
		\textbf{Κυβερνοασφάλεια}		& pfSense, PKI, OpenVPN, Python Scapy, iptables. \\
		\textbf{Εικονικοποίηση}			& Proxmox VE, QEMU/KVM, XCP-ng, VMware Workstation, VirtualBox, Docker.\\ 
		\textbf{DevOps}					& Docker, GitLab CI/CD. \\
		\textbf{Προγραματισμός}	    	& Python, JavaScript, Django, Java, MATLAB, C/C++. \\
		\textbf{Industrial IoT}			& Modbus TCP, MQTT, Raspberry Pi 4, OCPP 1.6-J.\\
		\textbf{Τηλεπικοινωνίες}   		& ITU-T PONs, Simulation (OMNeT++, GNS3). \\
		\textbf{Διάφορα}				& Αντιμετώπιση προβλημάτων, Διαχείριση έργων Horizon 2020, Σουΐτα Office, \LaTeX. \\
		\textbf{Γλώσσες} 				& Ελληνικά (εγγενώς), Αγγλικά (C2), Γερμανικά (C1).
	\end{tabular}

	\section{Εθελοντικες Δραστηριοτητες και Συνδρομες}
	\vspace*{0.125cm}	
	\begin{EntryDatedLogo}{Πανεπιστήμιο Δυτικής Μακεδονίας}{https://uowm.gr}{Μαρτ. 2016 - Τώρα}{Συντακτης εκπαιδευτικου υλικου}{-1.1cm}{uowm}{0.6}
		\begin{Itemize}
			\item Συγγραφή πρωτότυπου εκπαιδευτικού υλικού για το μάθημα «Σχεδίαση Δικτύων»: \url{https://github.com/cdalamagkas/network-design-labs}
			\item Η ύλη περιλαμβάνει δρομολόγηση και μεταγωγή με συσκευές Cisco και Mikrotik.
		\end{Itemize}
	\end{EntryDatedLogo}
	
	\vspace*{0.5cm}
	
	\begin{EntryDatedLogo}{Ινστιτούτο Ηλεκτρολόγων και Ηλεκτρονικών Μηχανικών}{https://www.ieee.org/}{Σεπτ. 2013 -- Τώρα}{Μελος}{-0.5cm}{ieee}{0.6}
	\end{EntryDatedLogo}
	
\end{document}
